\documentclass[11pt]{article}

\usepackage[headings]{fullpage}
\usepackage[utopia]{mathdesign}

\pagestyle{myheadings}
\markboth{MATH426/CISC410}{MATH426/CISC410}

\NeedsTeXFormat{LaTeX2e}
\ProvidesClass{fncextra}[2017/01/24 LaTeX Class For FNC extra materials]

\LoadClass{article}
\DeclareOption*{\PassOptionsToClass{\CurrentOption}{article}}
\ProcessOptions\relax

\RequirePackage{amsmath}
\RequirePackage[headings]{fullpage}
\RequirePackage[utopia]{mathdesign}
\RequirePackage{bm}
\RequirePackage{url}

\lstset{style=Matlab-editor,basicstyle=\ttfamily}
\definecolor{lightgray}{gray}{0.5}

\newcommand{\nat}{\mathbb{N}}          % Natural numbers
\newcommand{\integer}{\mathbb{Z}}      % Integers
\newcommand{\real}{ {\mathbb{R}} }     % Reals
\newcommand{\float}{ {\mathbb{F}} }     % Reals
\newcommand{\rmn}[2]{ \mathbb{R}^{#1\times#2} }     % Reals
\newcommand{\complex}{ {\mathbb{C}} }  % Complex
\newcommand{\macheps}{\ensuremath \varepsilon_{\text{mach}}}

\renewcommand{\Re}{\operatorname{Re}}
\renewcommand{\Im}{\operatorname{Im}}

% Boldface vectors
\newcommand{\bff}{\bm{f}}
\newcommand{\bfF}{\bm{F}}
\newcommand{\bfw}{\bm{w}}
\newcommand{\bfv}{\bm{v}}
\newcommand{\bfe}{\bm{e}}
\newcommand{\bfc}{\bm{c}}
\newcommand{\bfp}{\bm{p}}
\newcommand{\bfq}{\bm{q}}
\newcommand{\bfr}{\bm{r}}
\newcommand{\bfs}{\bm{s}}
\newcommand{\bfu}{\bm{u}}
\newcommand{\bfb}{\bm{b}}
\newcommand{\bfx}{\bm{x}}
\newcommand{\bfy}{\bm{y}}
\newcommand{\bfg}{\bm{g}}
\newcommand{\bfh}{\bm{h}}
\newcommand{\bfz}{\bm{z}}
\newcommand{\bfa}{\bm{a}}
\newcommand{\bft}{\bm{t}}
\newcommand{\bfd}{\bm{d}}
\newcommand{\bfalpha}{\bm{\alpha}}
\newcommand{\bfeps}{\bm{\varepsilon}}
\newcommand{\bfdelta}{\bm{\delta}}
\newcommand{\bfzero}{\bm{0}}
\newcommand{\eye}[1]{\bfe_{#1}}

% Boldface matrix
\newcommand{\m}[1]{\bm{#1}}
\newcommand{\mA}{\m{A}}
\newcommand{\mL}{\m{L}}
\newcommand{\mF}{\m{F}}
\newcommand{\mU}{\m{U}}
\newcommand{\mJ}{\m{J}}
\newcommand{\mP}{\m{P}}
\newcommand{\mQ}{\m{Q}}
\newcommand{\mR}{\m{R}}
\newcommand{\mD}{\m{D}}
\newcommand{\mS}{\m{S}}
\newcommand{\mB}{\m{B}}
\newcommand{\mC}{\m{C}}
\newcommand{\mE}{\m{E}}
\newcommand{\mG}{\m{G}}
\newcommand{\mH}{\m{H}}
\newcommand{\mV}{\m{V}}
\newcommand{\mW}{\m{W}}
\newcommand{\mX}{\m{X}}
\newcommand{\mZ}{\m{Z}}
\newcommand{\mK}{\m{K}}
\newcommand{\mM}{\m{M}}

\newcommand{\meye}{\m{I}}

\newcommand{\ee}[1]{\times 10^{#1}}
\newcommand{\jac}[2]{\frac{\bfd \bm{#1}}{\bfd \bm{#2}}}
\newcommand{\diag}{\operatorname{diag}}
\newcommand{\fl}{\operatorname{fl}}
\newcommand{\circop}[1]{\makebox[0pt][l]{$\bigcirc$}\hspace{1pt}#1}
\newcommand{\myvec}{\operatorname{vec}}
\newcommand{\unvec}{\operatorname{unvec}}
\newcommand{\kron}[2]{#1 \otimes #2}
\newcommand{\mtx}{\operatorname{mtx}}
\newcommand{\fun}{\operatorname{fun}}


\begin{document}

\begin{center}
  \bf Not entirely stable
\end{center}

Consider the function
\[
f(x) = \frac{e^x-1}{x}.
\]
A computation of its relative condition number $\kappa$ shows that $|\kappa(x)|<1$ for all $|x|<1$. In that sense, $f$ is ``easy'' to compute accurately. In practice, though, it's not so simple.

An obvious sequence of steps to compute $f$ is as follows:
\begin{equation}
  \label{direct}
  y_1 = e^x, \qquad y_2 = y_1 - 1, \qquad y_3 = y_2/x.
\end{equation}
The operations for $y_1$ and $y_3$ are well conditioned for $|x|<1$, but the subtraction to get $y_2$ will suffer from cancellation error if $y_1\approx 1$, or $x\approx 0$. That error makes this sequence of operations unstable for $x\approx 0$.

Now consider a power series expansion of $f$,
\begin{equation}
  \label{series}
  f(x) = 1 + \frac{1}{2!} x + \frac{1}{3!}x^2 + \cdots.
\end{equation}
For $x>0$, every term in the series is positive, leaving no possibility of cancellation error. (The analysis for negative $x$ is more subtle.) If $x\approx 0$, then we should be able to find $n$ such that $x^n/(n+1)!$ is smaller than machine precision, so that adding it to the terms before it will not change the result numerically. Thus a truncated form of the series can serve as a stable method for $x\approx 0$. 

\subsection*{Goals}

You will experiment with the two methods of computing $f$ and observe their relative accuracy. 

\subsection*{Preparation}

Read section 1.3.

\subsection*{Procedure}

Download the template script and complete it to perform the following steps. 

\begin{enumerate}
\item MATLAB has a stable way of computing $y_2$ in~\eqref{direct} without using subtraction. You will use it to get reference ``exact'' values of $f$. Let
\begin{verbatim}
x = logspace(-16,0,500);
\end{verbatim}
  Create a vector \texttt{y} the same size as \texttt{x} such that $y_j$ is the result of \texttt{expm1(x(j))/x(j)}.

\item Create a vector \texttt{z} such that $z_j$ is computed using the three steps in~\eqref{direct} for $x_j$.

\item Compute a vector of relative differences between $z_j$ and $y_j$. Make a log-log plot of the result as a function of $x$. You will see a loss of accuracy as $x\to 0$. 

\item By trial and error, find a value of $n$ such that $0.01^n/(n+1)!$ is less than \texttt{eps} (machine epsilon). This defines the last term to keep in truncating the series~\eqref{series}.

\item Create a vector \texttt{w} such that $w_j$ is computed using the truncated form of~\eqref{series}. Repeat step~3 for \texttt{w} in place of \texttt{z}, using \texttt{semilogx} instead of \texttt{loglog}. This time you should see accuracy maintained as $x\to 0$. (If more terms of the series were kept, the accuracy could be maintained as $x\to 1$ as well, but the direct method is more efficient there.)

\end{enumerate}


\end{document}
