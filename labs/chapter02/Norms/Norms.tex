\documentclass[11pt]{article}

\usepackage{amsmath,amsthm}
\usepackage[headings]{fullpage}
\usepackage[utopia]{mathdesign}
\usepackage{color}

\pagestyle{myheadings}
\markboth{Norms}{Norms}

\NeedsTeXFormat{LaTeX2e}
\ProvidesClass{fncextra}[2017/01/24 LaTeX Class For FNC extra materials]

\LoadClass{article}
\DeclareOption*{\PassOptionsToClass{\CurrentOption}{article}}
\ProcessOptions\relax

\RequirePackage{amsmath}
\RequirePackage[headings]{fullpage}
\RequirePackage[utopia]{mathdesign}
\RequirePackage{bm}
\RequirePackage{url}

\lstset{style=Matlab-editor,basicstyle=\ttfamily}
\definecolor{lightgray}{gray}{0.5}

\newcommand{\nat}{\mathbb{N}}          % Natural numbers
\newcommand{\integer}{\mathbb{Z}}      % Integers
\newcommand{\real}{ {\mathbb{R}} }     % Reals
\newcommand{\float}{ {\mathbb{F}} }     % Reals
\newcommand{\rmn}[2]{ \mathbb{R}^{#1\times#2} }     % Reals
\newcommand{\complex}{ {\mathbb{C}} }  % Complex
\newcommand{\macheps}{\ensuremath \varepsilon_{\text{mach}}}

\renewcommand{\Re}{\operatorname{Re}}
\renewcommand{\Im}{\operatorname{Im}}

% Boldface vectors
\newcommand{\bff}{\bm{f}}
\newcommand{\bfF}{\bm{F}}
\newcommand{\bfw}{\bm{w}}
\newcommand{\bfv}{\bm{v}}
\newcommand{\bfe}{\bm{e}}
\newcommand{\bfc}{\bm{c}}
\newcommand{\bfp}{\bm{p}}
\newcommand{\bfq}{\bm{q}}
\newcommand{\bfr}{\bm{r}}
\newcommand{\bfs}{\bm{s}}
\newcommand{\bfu}{\bm{u}}
\newcommand{\bfb}{\bm{b}}
\newcommand{\bfx}{\bm{x}}
\newcommand{\bfy}{\bm{y}}
\newcommand{\bfg}{\bm{g}}
\newcommand{\bfh}{\bm{h}}
\newcommand{\bfz}{\bm{z}}
\newcommand{\bfa}{\bm{a}}
\newcommand{\bft}{\bm{t}}
\newcommand{\bfd}{\bm{d}}
\newcommand{\bfalpha}{\bm{\alpha}}
\newcommand{\bfeps}{\bm{\varepsilon}}
\newcommand{\bfdelta}{\bm{\delta}}
\newcommand{\bfzero}{\bm{0}}
\newcommand{\eye}[1]{\bfe_{#1}}

% Boldface matrix
\newcommand{\m}[1]{\bm{#1}}
\newcommand{\mA}{\m{A}}
\newcommand{\mL}{\m{L}}
\newcommand{\mF}{\m{F}}
\newcommand{\mU}{\m{U}}
\newcommand{\mJ}{\m{J}}
\newcommand{\mP}{\m{P}}
\newcommand{\mQ}{\m{Q}}
\newcommand{\mR}{\m{R}}
\newcommand{\mD}{\m{D}}
\newcommand{\mS}{\m{S}}
\newcommand{\mB}{\m{B}}
\newcommand{\mC}{\m{C}}
\newcommand{\mE}{\m{E}}
\newcommand{\mG}{\m{G}}
\newcommand{\mH}{\m{H}}
\newcommand{\mV}{\m{V}}
\newcommand{\mW}{\m{W}}
\newcommand{\mX}{\m{X}}
\newcommand{\mZ}{\m{Z}}
\newcommand{\mK}{\m{K}}
\newcommand{\mM}{\m{M}}

\newcommand{\meye}{\m{I}}

\newcommand{\ee}[1]{\times 10^{#1}}
\newcommand{\jac}[2]{\frac{\bfd \bm{#1}}{\bfd \bm{#2}}}
\newcommand{\diag}{\operatorname{diag}}
\newcommand{\fl}{\operatorname{fl}}
\newcommand{\circop}[1]{\makebox[0pt][l]{$\bigcirc$}\hspace{1pt}#1}
\newcommand{\myvec}{\operatorname{vec}}
\newcommand{\unvec}{\operatorname{unvec}}
\newcommand{\kron}[2]{#1 \otimes #2}
\newcommand{\mtx}{\operatorname{mtx}}
\newcommand{\fun}{\operatorname{fun}}


\begin{document}

\begin{center}
  \bf Norm-an conquest
\end{center}

The absolute value $|x|$ represents the distance on the number line
between the point $x$ and the origin. We can do the same for a vector
of $n$ numbers, plotting an arrows whose tail is at the origin and
applying the Pythagorean theorem to conclude that $\bfx$ represents a
point that is a distance $(x_1^2 + \cdots + x_n^2)^{1/2}$ from the
origin. This is what we call the 2-norm, $\|\bfx\|_2$. It turns out
that if we abstract the notion of ``distance'' a little to its
essential properties, we can get other vector norms as well. Each norm has its own set of unit vectors (those having norm equal to 1).

For matrices, a definition based on the action of the
matrix as a linear transformation is quite handy for analyzing linear
algebra algorithms.  The definition starts with all the unit vectors
in whatever norm we like. Each of these
vectors $\bfv$ is mapped to the vector $\bfu=\mA\bfv$. The maximum of
$\|\bfu\|$ for all such $\bfv$ is the value of $\|\mA\|$. This
definition can be visualized easily only when both $\bfv$ and
$\bfu$ are low-dimensional.

\subsection*{Goals}

You will use experiments based on the definitions of norms to approximate the calculation of norms of a small matrix.

\subsection*{Preparation}

Read section 2.7. Recall that the equations $x=\cos(\theta)$, $y=\sin(\theta)$, $0\le \theta \le 2\pi$ parameterize the unit circle (set of all unit vectors) defined by the 2-norm in $\mathbf{R}^2$.

\subsection*{Procedure}

Download the script template and complete it to perform the following steps.

\begin{enumerate}
\item Define a vector \texttt{theta} of 200 equally spaced values from 0 to $2\pi$. Use it to define vectors \texttt{x} and \texttt{y} via $x_j = \cos(\theta_j)$, $y_j = \sin(\theta_j)$ (do this without any loops). Plot these points, which lie on the unit circle.

\item Let \texttt{A=magic(2)}. For each $j$, define the vector $\bfv=[x_j\:y_j]^T$ and let $\bfu=\mA\bfv$. On a new graph, plot all the $\bfu$ points, and make a vector storing $\|\bfu\|_2$ for all $j$.

\item Plot $\|\bfu\|_2$ as a function of $\theta$. 

\item Using \texttt{format long} and \texttt{max}, calculate and show the maximum value of $\|\bfu\|_2$ over all $j$. This estimates $\|\mA\|_2$. Also compute the actual value \texttt{norm(A,2)}.
  
\item The script creates new vectors \texttt{x} and \texttt{y} that are unit vectors in the $\infty$-norm. Plot these points, giving a title and axis labels.

\item Repeat step 2, this time collecting $\|\bfu\|_\infty$ as a function of the point index $j$.
  
\item Make a plot as in step 3. This time, the $x$-axis will the index number $j$.

\item Estimate the norm $\|\mA\|_\infty$ and compute its exact value, as in step 4. 
\end{enumerate}

\end{document}

%%% Local Variables:
%%% mode: latex
%%% TeX-master: t
%%% End:
