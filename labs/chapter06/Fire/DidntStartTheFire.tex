\documentclass[twoside]{fncextra}

\usepackage{color}
\usepackage{graphicx}
\usepackage[pdftex]{hyperref}

\pagestyle{myheadings}
\markboth{We didn't start the fire}{MATH426/CISC410}

\begin{document}

\begin{center}
  \bf We didn't start the fire
\end{center}

The initial-value problem
\begin{equation}
  \label{eq:8}
  \frac{dr}{dt} = r^2(1-r), \qquad t > 0, \quad r(0)=r_0,
\end{equation}
is a simple model for the radius of a \href{https://youtu.be/Q58-la_yAB4}{spherical flame ball in zero gravity}. As $t\to\infty$, the solution tends to 1. In MATLAB, you can create a high-accuracy reference solution by entering, for example:
\begin{verbatim}
f = @(t,r) r^2*(1-r);
opt = odeset('reltol',1e-13,'abstol',1e-13);
soln = ode113(f,[0 500],0.01,opt);
r_hat = @(t) deval(soln,t);
\end{verbatim}
The second line sets an absolute and relative (whichever is weaker) error target for the solution. The third line causes MATLAB to find all the information needed for a numerical solution with $r(0)=0.01$ over $0\le t \le 500$ and returns the data in a special structure. The last line creates a callable function that uses the data structure. 
Thereafter, you can evaluate the reference solution at any time $t$ just by calling \verb!r_hat(t)!, even if the \texttt{soln} variable is later changed.

\subsection*{Preparation}

Read Section 6.2. On one labeled plot, graph the reference solution described above, $\hat{r}(t)$, for $0\le t\le 500$ for $r_0=0.05$, 0.01, and 0.005. 

\subsection*{Goals}

Although the Euler method (as implemented by \texttt{eulerivp} in the book) is first-order convergent, the behavior of its solutions can be surprisingly complicated, as these experiments will reveal.


\subsection*{Procedure}
\label{sec:procedure}

\begin{enumerate}
  \item Compute the reference solution for $r_0=0.01$ over $0\le t\le 500$, and plot it. 
  \item Use \texttt{eulerivp} to compute the solution over $0\le t \le 100$ with step size $h=1$. Plot the reference solution and the Euler solution together over $[0,100]$.
  \item For $n = 50,100,150,\ldots,10000$, use the \texttt{eulerivp} function to solve the IVP with $r_0=0.01$ up to time $t=100$. Make a log-log plot of the error at time $t=100$ versus $n$, and verify that the method is first-order convergent.
  \item Repeat part 3 for solving at time $t=500$. The convergence behavior is \emph{very} different from the previous case.
  \item In a $4\times 1$ array of subplots, graph the solution $r(t)$ over $0\le t \le 500$ using \texttt{eulerivp} to solve with $n=100$, 150, 200, and 250. As you can see, the size of the error at a single time does not tell the full story of how the numerical solutions behave.
%  \item The essence of what is happening here is captured by the even simpler problem $r'=-r$. Solve this one under the same conditions as in part (4), and the similarities should become clear. 
\end{enumerate}

\end{document}
