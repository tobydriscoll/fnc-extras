\documentclass[11pt,twoside]{article}

\usepackage[headings]{fullpage}
\usepackage{hyperref}
\pagestyle{myheadings}
\markboth{Fire}{Fire}

\usepackage[utopia]{mathdesign}

\NeedsTeXFormat{LaTeX2e}
\ProvidesClass{fncextra}[2017/01/24 LaTeX Class For FNC extra materials]

\LoadClass{article}
\DeclareOption*{\PassOptionsToClass{\CurrentOption}{article}}
\ProcessOptions\relax

\RequirePackage{amsmath}
\RequirePackage[headings]{fullpage}
\RequirePackage[utopia]{mathdesign}
\RequirePackage{bm}
\RequirePackage{url}

\lstset{style=Matlab-editor,basicstyle=\ttfamily}
\definecolor{lightgray}{gray}{0.5}

\newcommand{\nat}{\mathbb{N}}          % Natural numbers
\newcommand{\integer}{\mathbb{Z}}      % Integers
\newcommand{\real}{ {\mathbb{R}} }     % Reals
\newcommand{\float}{ {\mathbb{F}} }     % Reals
\newcommand{\rmn}[2]{ \mathbb{R}^{#1\times#2} }     % Reals
\newcommand{\complex}{ {\mathbb{C}} }  % Complex
\newcommand{\macheps}{\ensuremath \varepsilon_{\text{mach}}}

\renewcommand{\Re}{\operatorname{Re}}
\renewcommand{\Im}{\operatorname{Im}}

% Boldface vectors
\newcommand{\bff}{\bm{f}}
\newcommand{\bfF}{\bm{F}}
\newcommand{\bfw}{\bm{w}}
\newcommand{\bfv}{\bm{v}}
\newcommand{\bfe}{\bm{e}}
\newcommand{\bfc}{\bm{c}}
\newcommand{\bfp}{\bm{p}}
\newcommand{\bfq}{\bm{q}}
\newcommand{\bfr}{\bm{r}}
\newcommand{\bfs}{\bm{s}}
\newcommand{\bfu}{\bm{u}}
\newcommand{\bfb}{\bm{b}}
\newcommand{\bfx}{\bm{x}}
\newcommand{\bfy}{\bm{y}}
\newcommand{\bfg}{\bm{g}}
\newcommand{\bfh}{\bm{h}}
\newcommand{\bfz}{\bm{z}}
\newcommand{\bfa}{\bm{a}}
\newcommand{\bft}{\bm{t}}
\newcommand{\bfd}{\bm{d}}
\newcommand{\bfalpha}{\bm{\alpha}}
\newcommand{\bfeps}{\bm{\varepsilon}}
\newcommand{\bfdelta}{\bm{\delta}}
\newcommand{\bfzero}{\bm{0}}
\newcommand{\eye}[1]{\bfe_{#1}}

% Boldface matrix
\newcommand{\m}[1]{\bm{#1}}
\newcommand{\mA}{\m{A}}
\newcommand{\mL}{\m{L}}
\newcommand{\mF}{\m{F}}
\newcommand{\mU}{\m{U}}
\newcommand{\mJ}{\m{J}}
\newcommand{\mP}{\m{P}}
\newcommand{\mQ}{\m{Q}}
\newcommand{\mR}{\m{R}}
\newcommand{\mD}{\m{D}}
\newcommand{\mS}{\m{S}}
\newcommand{\mB}{\m{B}}
\newcommand{\mC}{\m{C}}
\newcommand{\mE}{\m{E}}
\newcommand{\mG}{\m{G}}
\newcommand{\mH}{\m{H}}
\newcommand{\mV}{\m{V}}
\newcommand{\mW}{\m{W}}
\newcommand{\mX}{\m{X}}
\newcommand{\mZ}{\m{Z}}
\newcommand{\mK}{\m{K}}
\newcommand{\mM}{\m{M}}

\newcommand{\meye}{\m{I}}

\newcommand{\ee}[1]{\times 10^{#1}}
\newcommand{\jac}[2]{\frac{\bfd \bm{#1}}{\bfd \bm{#2}}}
\newcommand{\diag}{\operatorname{diag}}
\newcommand{\fl}{\operatorname{fl}}
\newcommand{\circop}[1]{\makebox[0pt][l]{$\bigcirc$}\hspace{1pt}#1}
\newcommand{\myvec}{\operatorname{vec}}
\newcommand{\unvec}{\operatorname{unvec}}
\newcommand{\kron}[2]{#1 \otimes #2}
\newcommand{\mtx}{\operatorname{mtx}}
\newcommand{\fun}{\operatorname{fun}}


\begin{document}

\begin{center}
  \bf We didn't start the fire
\end{center}

The initial-value problem
\begin{equation}
  \label{eq:fire}
  \frac{dr}{dt} = r^2(1-r), \qquad t > 0, \quad r(0)=r_0,
\end{equation}
is a simple model for the radius of a \href{https://youtu.be/Q58-la_yAB4}{spherical flame ball in zero gravity}. If $r$ is initially small, it grows slowly for a while before rapidly increasing to a value close to 1, which it approaches asymptotically. 

Although the solution is quite simple, it proves to be surprisingly challenging for some IVP solvers, including Euler's method.

\subsection*{Preparation}

Read sections 6.1 and 6.2. 

\subsection*{Goals}

You will solve the flame ball equation numerically using Euler's method, observing its convergence. 

\subsection*{Procedure}

Download the template code and edit it to perform the following steps. 

\begin{enumerate}
	\item Follow the script to create a high-accuracy reference solution of the problem with $r_0=0.01$. Plot the reference solution for $0\le t\le 500$. 
  \item For $n = 100,200,400,800,1600,6400,25600$, use the
    \texttt{eulerivp} function to solve the IVP with $r_0=0.01$ up to
    time $t=100$. Find the error in the Euler solution by taking the
    difference between its value of $r$ at
   and the reference solution at $t=100$. Make a log-log plot of the errors and verify that the method is first-order convergent.
  \item Repeat part 2 but at time $t=200$. This time, the convergence behavior is a lot less smooth than first-order accuracy might seem to imply.
  \item Plot the solution $r(t)$ over $0\le t \le 500$ using \texttt{eulerivp} with $n=150$, 200, 250, and 300. Even though the errors are getting smaller, these could hardly be called ``good'' solutions qualitatively!
\end{enumerate}


\end{document}
