\documentclass[11pt,twoside]{article}

\usepackage[headings]{fullpage}
\usepackage[utopia]{mathdesign}

\pagestyle{myheadings}
\markboth{What goes around}{What goes around}

\NeedsTeXFormat{LaTeX2e}
\ProvidesClass{fncextra}[2017/01/24 LaTeX Class For FNC extra materials]

\LoadClass{article}
\DeclareOption*{\PassOptionsToClass{\CurrentOption}{article}}
\ProcessOptions\relax

\RequirePackage{amsmath}
\RequirePackage[headings]{fullpage}
\RequirePackage[utopia]{mathdesign}
\RequirePackage{bm}
\RequirePackage{url}

\lstset{style=Matlab-editor,basicstyle=\ttfamily}
\definecolor{lightgray}{gray}{0.5}

\newcommand{\nat}{\mathbb{N}}          % Natural numbers
\newcommand{\integer}{\mathbb{Z}}      % Integers
\newcommand{\real}{ {\mathbb{R}} }     % Reals
\newcommand{\float}{ {\mathbb{F}} }     % Reals
\newcommand{\rmn}[2]{ \mathbb{R}^{#1\times#2} }     % Reals
\newcommand{\complex}{ {\mathbb{C}} }  % Complex
\newcommand{\macheps}{\ensuremath \varepsilon_{\text{mach}}}

\renewcommand{\Re}{\operatorname{Re}}
\renewcommand{\Im}{\operatorname{Im}}

% Boldface vectors
\newcommand{\bff}{\bm{f}}
\newcommand{\bfF}{\bm{F}}
\newcommand{\bfw}{\bm{w}}
\newcommand{\bfv}{\bm{v}}
\newcommand{\bfe}{\bm{e}}
\newcommand{\bfc}{\bm{c}}
\newcommand{\bfp}{\bm{p}}
\newcommand{\bfq}{\bm{q}}
\newcommand{\bfr}{\bm{r}}
\newcommand{\bfs}{\bm{s}}
\newcommand{\bfu}{\bm{u}}
\newcommand{\bfb}{\bm{b}}
\newcommand{\bfx}{\bm{x}}
\newcommand{\bfy}{\bm{y}}
\newcommand{\bfg}{\bm{g}}
\newcommand{\bfh}{\bm{h}}
\newcommand{\bfz}{\bm{z}}
\newcommand{\bfa}{\bm{a}}
\newcommand{\bft}{\bm{t}}
\newcommand{\bfd}{\bm{d}}
\newcommand{\bfalpha}{\bm{\alpha}}
\newcommand{\bfeps}{\bm{\varepsilon}}
\newcommand{\bfdelta}{\bm{\delta}}
\newcommand{\bfzero}{\bm{0}}
\newcommand{\eye}[1]{\bfe_{#1}}

% Boldface matrix
\newcommand{\m}[1]{\bm{#1}}
\newcommand{\mA}{\m{A}}
\newcommand{\mL}{\m{L}}
\newcommand{\mF}{\m{F}}
\newcommand{\mU}{\m{U}}
\newcommand{\mJ}{\m{J}}
\newcommand{\mP}{\m{P}}
\newcommand{\mQ}{\m{Q}}
\newcommand{\mR}{\m{R}}
\newcommand{\mD}{\m{D}}
\newcommand{\mS}{\m{S}}
\newcommand{\mB}{\m{B}}
\newcommand{\mC}{\m{C}}
\newcommand{\mE}{\m{E}}
\newcommand{\mG}{\m{G}}
\newcommand{\mH}{\m{H}}
\newcommand{\mV}{\m{V}}
\newcommand{\mW}{\m{W}}
\newcommand{\mX}{\m{X}}
\newcommand{\mZ}{\m{Z}}
\newcommand{\mK}{\m{K}}
\newcommand{\mM}{\m{M}}

\newcommand{\meye}{\m{I}}

\newcommand{\ee}[1]{\times 10^{#1}}
\newcommand{\jac}[2]{\frac{\bfd \bm{#1}}{\bfd \bm{#2}}}
\newcommand{\diag}{\operatorname{diag}}
\newcommand{\fl}{\operatorname{fl}}
\newcommand{\circop}[1]{\makebox[0pt][l]{$\bigcirc$}\hspace{1pt}#1}
\newcommand{\myvec}{\operatorname{vec}}
\newcommand{\unvec}{\operatorname{unvec}}
\newcommand{\kron}[2]{#1 \otimes #2}
\newcommand{\mtx}{\operatorname{mtx}}
\newcommand{\fun}{\operatorname{fun}}


\begin{document}

\begin{center}
  \bf What goes around, comes around
\end{center}


A small satellite in the Earth--moon system can be modeled as a \emph{restricted three-body problem}, in which one object has so little relative mass that the motions essentially take place in a plane. In this formulation, Newton's laws of motion are simplified by a transformation so that the Earth has mass $\mu_*$ and remains at $(0,0)$, while the moon has mass $\mu$ and is at $(1,0)$, where $\mu+\mu_*=1$. The position of the satellite is $(x(t),y(t))$ and satisfies
\begin{equation}
\begin{split}
  x'' &= x + 2y' - \mu_* \frac{x+\mu}{r} - \mu \frac{x-\mu_*}{r_*} \\
  y'' &= y - 2x' - \mu_* \frac{y}{r} - \mu \frac{y}{r_*}, 
\end{split}\label{eq:r3body}
\end{equation}
where
\begin{align*}
  r &= \left[ (x+\mu)^2 + y^2 \right]^{3/2} \\
	r_* &= \left[ (x-\mu_*)^2 + y^2 \right]^{3/2}.
\end{align*}
For the Earth/moon system, $\mu=0.012277471$.

The key to computing solutions using standard software is to transform the original ODE system into a first-order one. Since there are two dependent variables $x$ and $y$, and both of these appear to second order, there will be 4 variables and 4 equations in the first-order version. Define $u_1=x$, $u_2=y$, $u_3=x'$, $u_4=y'$. Two trivial equations in the new system are $u_1'=u_3$ and $u_2'=u_4$. The other two equations, for $u_3'$ and $u_4'$, come from substitution into the original system~\eqref{eq:r3body}. For example,
\begin{equation*}
  u_3' = u_1 + 2u_4 - \mu_* \frac{u_1+\mu}{r} - \mu \frac{u_1-\mu_*}{r_*},
\end{equation*}
and so on.

\subsection*{Goals}

You will explore unlikely-looking satellite orbits that have fascinated mathematicians since Poincar\'e. They are of more than academic interest, because despite their strangeness, they are energy-efficient. 

\subsection*{Preparation}

Read section 6.1. Fully write out the first-order system of ODEs $\bfu'=\bff(t,\bfu)$ that is equivalent to~\eqref{eq:r3body}. 

\subsection*{Procedure}

Download the script template and complete it to execute the following steps. 

\begin{enumerate}
\item Set
\begin{verbatim}
opt = odeset('reltol',1e-13,'abstol',1e-13);
\end{verbatim}
This will be used below to require strict error tolerances in the ODE solutions. The solutions below will all use \texttt{ode113} with \texttt{opt} as a fourth input argument.

\item Using the expression of equation~\eqref{eq:r3body} as a first-order system, $\bfu'=\bff\,(t,\bfu)$, write the function
\begin{verbatim}
function dudt = r3body(t,u)
\end{verbatim}
in which \texttt{u} is a vector containing numerical values for the dependent variables, and \texttt{dudt} returns the components of $\bfu'$. 

\item Using the initial conditions 
  \begin{align*}
    x(0) &= 1.2, \; & x'(0) &= 0, \\
    y(0) &= 0, \; & y'(0) &= -1.049357510,
  \end{align*}
solve the problem on the interval $[0,6.192169331]$, using \texttt{ode113} and passing \texttt{opt} as a fourth input argument. Plot the trajectory in the $(x,y)$ phase plane. It will have two major loops, one to each side of the Earth. Add the moon and Earth positions as points in the plot.  

\item Repeat step 3 for
  \begin{align*}
    x(0) &= 0.994 \; & x'(0) &= 0 \\
    y(0) &= 0 \; & y'(0) &= -2.03173262955734
  \end{align*}
  and solve over $[0,11.124340337266]$. This orbit has three loops.

\item Repeat step 3 for 
      \begin{align*}
        x(0) &= 0.994 \; & x'(0) &= 0 \\
        y(0) &= 0 \; & y'(0) &= -2.00158510637908
      \end{align*}
      and solve over $[0,17.06521656015796]$. This one has four loops. 

\item These orbits are very sensitive to perturbations, and while our convergence theorems apply to a fixed time interval and $h\to 0$, the case of $t\to\infty$ is quite different. Re-solve step~5 for $0\le t \le 100$. Your plot will show the path of the satellite departing from the apparently periodic orbit. 
\end{enumerate}

\subsection*{Discussion}

In step 3 above, what is the maximum speed of the satellite over the whole orbit? 


\end{document}

%%% Local Variables: 
%%% mode: latex
%%% TeX-master: t
%%% End: 
